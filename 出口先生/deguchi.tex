\documentclass[10pt]{jreport}
\usepackage[dvipdfmx]{graphicx}
\usepackage{cases}


\begin{document}
\title{理学総論レポート(担当:出口先生)}

\author{お茶の水女子大学人間文化創成科学研究科理学専攻 \\1940620 \\ 前田実津季}
\date{\today}
\maketitle

\section*{揺動散逸定理の一種を導出する}
今、小さい粒子が水中を動いている状況を考える。\\
水の粘性、水中に粘性力をFとおく。\\
このとき、ストークスの法則より、粘性力は、次のように表すことができる。\\
\begin{equation}
F=6\pi \eta av
\end{equation}
この時の、$\eta$は、粘性係数である。また、aは小さい粒子の半径、Vは小さい粒子の速度である。\\
ここで、$\xi = 6\pi\eta a$をおくと、(1)式は、
\begin{equation}
F = \xi v
\end{equation}
となる。\\
この粘性力に加えて、水中では小さい粒子は水分子の熱運動によって乱雑な力が働く。(R(t))\\
次の方程式をランジュバン方程式と呼び、
\begin{equation}
m\frac{dv}{dt}=-\xi v +R(t)
\end{equation}
このランダム力の平均値は0である。
\begin{equation}
\langle R(t)\rangle = 0
\end{equation}
また、このランダム力の分散は、
\begin{equation}
\langle R(t_1)R(t_2) \rangle = 2\varepsilon \delta(t_1-t_2)
\end{equation}
となり、同時刻のランダム力の分散の平均値は(5)式から$2\varepsilon$となることがわかる。\\
この$\varepsilon$について、これから$\varepsilon=\xi k_B T$となることを示す。\\

粘性係数が非常に大きい粘性極限では、(3)式は、
\begin{equation}
m\frac{dv}{dt}=0
\end{equation}
となり、
\begin{eqnarray}
-\xi v+R(t) =0 \nonumber \\
\Leftrightarrow \frac{dx}{dt}=\frac{1}{\xi}R(t)
\end{eqnarray}
となる。\\
$x(t)-x(0)=\int^t_0 \frac{dx}{dt}$であるので、
\begin{eqnarray}
\langle (x(t)-x(0))^2\rangle &=& \langle \int^t_0\frac{1}{\xi}R(t_1)dt_1\int^t_0\frac{1}{\xi}R(t_2)dt_2\rangle \nonumber \\
&=& \frac{1}{\xi ^2} \int^t_0dt_1 \int^t_0dt_2\langle R(t_1)R(t_2)\rangle \nonumber \\
&=& \frac{2\varepsilon}{\xi ^2}\int^t_0dt_1\nonumber \\
&=& \frac{2\varepsilon}{\xi^2} t 
\end{eqnarray}
となる。\\
ここで、$\varepsilon=\xi k_B T$を仮定する。\\
とすると、
\begin{equation}
\langle (x(t)-x(0))^2 \rangle =2 \frac{k_BT}{\xi}t
\end{equation}
となり、つまり、時間とともに、粒子の位置がずれていくということである。Dを拡散係数とおくと、
\begin{equation}
D=\frac{k_BT}{\xi}
\end{equation}
となり、これはアインシュタインの関係式を呼ばれる。この関係式から、$\xi = 6\pi \eta a$であり、Dとaは反比例の関係にあることもわかる。\\
(3)式の
\begin{eqnarray}
m\frac{dv}{dt}=-\xi v +R(t) \nonumber
\end{eqnarray}
は、線形の非斉次方程式なので、定数変形法で解くことができる。
\begin{equation}
m\frac{dv}{dt}=-\xi v
\end{equation}
変数分離法により、
\begin{eqnarray}
mdv = -\xi v dt \nonumber \\
\frac{dv}{v} = -\frac{\xi}{m} dt 
\end{eqnarray}
これを両辺積分して、
\begin{eqnarray}
\log v = -\frac{\xi}{m}t + c\nonumber \\
v = c\exp(-\frac{\xi}{m}t)
\end{eqnarray}
積分定数のcを時間に依存するとし、$c \rightarrow c(t)$と仮定する。すると、(13)式は、
\begin{equation}
v(t) = c(t)e^{-\frac{\xi}{m}t}
\end{equation}
となる。時間で微分すると、
\begin{equation}
\frac{dv}{dt}=-\frac{\xi}{m}e^{-\frac{\xi}{m}t}c(t) + \frac{dc}{dt}e^{-\frac{\xi}{m}t}
\end{equation}
となる。両辺にmをかけ、(3)式を比較すると、
\begin{eqnarray}
m\frac{dv}{dt} &=& -\xi c(t)e^{-\frac{\xi}{m}t}+m\frac{dc}{dt}e^{-\frac{\xi}{m}t} \\
&=& -\xi v +R(t)
\end{eqnarray}
つまり、
\begin{eqnarray}
\left \{
\begin{array}{l}
v = c(t) e^{-\frac{\xi}{m}t} \\
R(t) =m\frac{dc}{Dt}e^{-\frac{\xi}{m}t}
\end{array}
\right.
\end{eqnarray}
となる。よって、
\begin{eqnarray}
\frac{dc}{dt} &=& \frac{1}{m}R(t)e^{\frac{\xi}{m}t} \nonumber \\
c(t) &=& \frac{1}{m}\int^t_0R(t_1)e^{\frac{\xi}{m}t_1}dt_1 +c_0
\end{eqnarray}
とおける。この$c(t)$を(14)式に代入すると、
\begin{eqnarray}
v(t) = e^{-\frac{\xi}{m}t}\int^t_0 \frac{1}{m}R(\tau)e^{\frac{\xi}{m}\tau}d\tau +c_0e^{-\frac{\xi}{m}t}
\end{eqnarray}
となる。これを用いて、$\langle v(t_1)v(t_2) \rangle$を計算する。\\
(4)式から、$\langle R(t)\rangle =0 $より、$c_0$の項は、時間が十分にたてば消えるので、この項は無視して、計算を行うことにする。
\begin{eqnarray}
\langle v(t_1)v(t_2) \rangle &=& \langle e^{-\frac{\xi}{m}t_1}\int^{t_1}_0 \frac{1}{m} rR(\tau)e^{\frac{\xi}{m}\tau}d\tau e^{-\frac{\xi}{m}t_2}\int^{t_2}_0\frac{1}{m}R(\tau ')e^{\frac{\xi}{m}\tau '}d\tau ' \rangle \nonumber \\
&=& \frac{1}{m^2} e^{-\frac{\xi}{m}(t_1+t_2)}\int^{t_1}_0d\tau \int^{t_2}_0d\tau ' e^{\frac{\xi}{m}(\tau+\tau')}\langle R(\tau)R(\tau') \rangle
\end{eqnarray}
ここで$\langle R(\tau R(\tau ') \rangle=2\varepsilon \delta(\tau-\tau ')$で、今、$t_1 > t_2$のとき、(21)式は
\begin{eqnarray}
&=& \frac{1}{m^2}e^{-\frac{\xi}{m}t_1+t_2}\int^{t_1}_0 d\tau e^{-\frac{\xi}{m}\tau}2\varepsilon \nonumber \\
&=& \frac{1}{m^2}e^{-\frac{\xi}{m}t_1+t_2}(\frac{m}{2\xi})[e^{-\frac{2\xi}{m}\tau}]^{t_1}_0 2\varepsilon \nonumber \\
&=& \frac{\varepsilon}{m\xi} e^{-\frac{\xi}{m}t_1+t_2}(e^{-\frac{2\xi}{m}t_1}-1) \nonumber \\
&=& \frac{\varepsilon}{m\xi} (e^{-\frac{\xi}{m}(-t_1+t_2)}-e^{-\frac{\xi}{m}(t_1+t_2)})
\end{eqnarray}
今仮定で、$t_1 > t_2$のときを考えたが、逆もあり得るので、一般的に、$\langle v(t_1)v(t_2) \rangle$は、
\begin{equation}
\underline{\langle v(t_1)v(t_2) \rangle =\frac{\varepsilon}{m\xi} (e^{-\frac{\xi}{m}\|t_1-t_2\|}-e^{-\frac{\xi}{m}(t_1+t_2)})}
\end{equation}
となる。\\
今、$t_1,t_2 \gg 1$でかつ、$t_1=t_2$のとき、
\begin{equation}
\langle v(t_1)^2 \rangle \simeq \frac{\varepsilon}{m\xi}
\end{equation}
で、エネルギー等分配則より、
\begin{equation}
\frac{m}{2} \langle v(t)^2 \rangle = \frac{1}{2}k_BT
\end{equation}
よって、
\begin{eqnarray}
\frac{\varepsilon}{2\xi} = \frac{1}{2}k_BT \nonumber \\
\underline{\varepsilon =\xi k_BT }
\end{eqnarray}
となることがわかった。

\end{document}